\documentclass{article}

\usepackage[a4paper, margin=2cm]{geometry}
\usepackage{microtype}
\usepackage{amsmath}
\usepackage{amssymb}
\usepackage{enumitem}
\usepackage{array}
\usepackage{tikz}
\usetikzlibrary{cd}

\newcommand\quotient[2]{{^{\displaystyle #1}}/{_{\displaystyle #2}}}

\begin{document}
\begin{enumerate}[style=nextline,label={Problem (\arabic*)}]
\item {
\begin{enumerate}[label={(\alph*)}]
\item {
Let $P = \{I \subsetneq R \mid I \text{ is an ideal of } R\}$ be the partially ordered set of proper ideals of $R$. Then $\mathfrak{m} \in P$ is called a maximal ideal if it is a maximal element of this partially ordered set.
\par
Equivalently, this means that $\quotient{R}{\mathfrak{m}}$ is a field.
}
\item {
Let $a \not\in \mathfrak{m}$. Then $\langle a \rangle + \mathfrak{m} \supsetneq \mathfrak{m}$, so $\langle a \rangle + \mathfrak{m} = R$. Hence there is $b \in R$ and $c \in \mathfrak{m}$ with $ab + c = 1$, so that $ab = 1 - c$. But now $1 - c \in 1 + \mathfrak{m}$, which by assumption only consists of units. Hence $ab$ is a unit. But then both factors must be units, so that $a$ must be a unit.
\par
We have shown that $R \setminus \mathfrak{m} \subseteq R^\times$, i.e. $\mathfrak{m} \supseteq R \setminus R^\times$. But any proper ideal only consists of non-units, so that $\mathfrak{m} \subseteq R \setminus R^\times$. Hence $\mathfrak{m} = R \setminus R^\times$, which means exactly that $R$ is local with unique maximal ideal $\mathfrak{m}$.
}
\item {
Let $\overline{\mathfrak{m}} \subseteq \quotient{\mathbb{Q}[x, y]}{\langle x^{20}, y^{20} \rangle}$ be a maximal ideal. This corresponds to a maximal ideal $\mathfrak{m} \subseteq \mathbb{Q}[x, y]$ with $\langle x^{20}, y^{20} \rangle \subseteq \mathfrak{m}$. Hence $x^{20}, y^{20} \in \mathfrak{m}$. But since $\mathfrak{m}$ is a maximal ideal, it's also a prime ideal. Hence $x, y \in \mathfrak{m}$, hence $\langle x, y \rangle \subseteq \mathfrak{m}$. Since $\langle x, y \rangle \subseteq \mathbb{Q}[x, y]$ is a maximal ideal, we have $\langle x, y \rangle = \mathfrak{m}$. Hence $\quotient{\mathbb{Q}[x, y]}{\langle x^{20}, y^{20} \rangle}$ has a unique maximal ideal, i.e. it's a local ring.
}
\item {
One can see that the map
\[
\begin{array}{>{\displaystyle}c}
\quotient{\mathbb{C}[x, y]}{\langle x^3 - y^5 \rangle} \to \mathbb{C}[t^3, t^5], \\ \relax
[x] \mapsto t^5, \\ \relax
[y] \mapsto t^3 \\
\end{array}
\]
is an isomorphism.
\par
Clearly, $\mathbb{C}[t^3, t^5]$ is an integral domain but not a field. Hence $\langle x^3 - y^5 \rangle \subseteq \mathbb{C}[x, y]$ is a prime ideal but not a maximal ideal.
}
\end{enumerate}
}
\item {
\begin{enumerate}[label={(\alph*)}]
\item {
Let $M$ and $N$ be $R$-modules. Then the tensor product $M \otimes_R N$ is an $R$-module equipped with a bilinear map $\alpha: M \times N \to M \otimes_R N$ such that for any $R$-module $P$ and any bilinear map $\beta: M \times N \to P$ there is a unique linear map $\tilde{\beta}: M \otimes_R N \to P$ with $\beta = \tilde{\beta} \circ \alpha$.
\begin{center}
% https://tikzcd.yichuanshen.de/#N4Igdg9gJgpgziAXAbVABwnAlgFyxMJZABgBpiBdUkANwEMAbAVxiRAFkACAHW7wFt4nAHIgAvqXSZc+QigCM5KrUYs2XXhAHwA+gCUR4ySAzY8BIovnL6zVohAAFccphQA5vCKgAZgCcIfiQyEBwIJEUVOzZeRjQACzojXwCgxBCwpAAmals1B14AIxgcJOoGOmKGR2lzORA-LHd4nGSQf0CI6kzEHKj8kF48BlhgIpK6MRcxIA
\begin{tikzcd}
M \times N \arrow[r, "\alpha"] \arrow[rd, "\beta"'] & M \otimes_R N \arrow[d, "\tilde{\beta}"] \\
                                                    & P                                       
\end{tikzcd}
\end{center}
}
\item {
Take any element of $U^{-1} R \otimes_R M$. Such an element can be written as $\sum_{i = 1}^n \frac{a_i}{b_i} c_i$ with $a_i \in R$, $b_i \in U$ and $c_i \in M$. But now
\[
\sum_{i = 1}^n \frac{a_i}{b_i} c_i = \sum_{i = 1}^n \frac{\left( \prod_{j \neq i} b_j \right) a_i}{\prod_{j} b_j} c_i = \frac{1}{\prod_{j} b_j} \sum_{i = 1}^n \left( \prod_{j \neq i} b_j \right) a_i c_i
\]
with $\prod_{j} b_j \in U$ and $\sum_{i = 1}^n \left( \prod_{j \neq i} b_j \right) a_i c_i \in M$.
}
\item {
We have
\[
\begin{array}{>{\displaystyle}c>{\displaystyle}l}
& \mathbb{Q} \otimes_{\mathbb{Z}} \left( \mathbb{Z} \oplus \quotient{\mathbb{Z}}{\langle 42 \rangle} \right) \\
\cong & \left( \mathbb{Q} \otimes_{\mathbb{Z}} \mathbb{Z} \right) \oplus \left( \mathbb{Q} \otimes_{\mathbb{Z}} \quotient{\mathbb{Z}}{\langle 42 \rangle} \right) \\
\cong & \mathbb{Q} \oplus 0 \\
\cong & \mathbb{Q} \\
\end{array}
\]
}
\item {
}
\item {
}
\end{enumerate}
}
\item {
\begin{enumerate}[label={(\alph*)}]
\item {
}
\item {
}
\item {
}
\item {
}
\item {
}
\end{enumerate}
}
\end{enumerate}
\end{document}