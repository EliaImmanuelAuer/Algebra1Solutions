\documentclass{article}

\usepackage[a4paper, margin=2cm]{geometry}
\usepackage{microtype}
\usepackage{amsmath}
\usepackage{amssymb}
\usepackage{enumitem}
\usepackage{array}
\usepackage{tikz}
\usetikzlibrary{cd}

\newcommand\quotient[2]{{^{\displaystyle #1}}/{_{\displaystyle #2}}}

\begin{document}
\begin{enumerate}[style=nextline,label={Problem (\arabic*)}]
\item {
\begin{enumerate}[label={(\alph*)}]
\item {
Let $P = \{I \subsetneq R \mid I \text{ is an ideal of } R\}$ be the partially ordered set of proper ideals of $R$. Then $\mathfrak{m} \in P$ is called a maximal ideal if it is a maximal element of this partially ordered set.
\par
Equivalently, this means that $\quotient{R}{\mathfrak{m}}$ is a field.
}
\item {
Let $a \not\in \mathfrak{m}$. Then $\langle a \rangle + \mathfrak{m} \supsetneq \mathfrak{m}$, so $\langle a \rangle + \mathfrak{m} = R$. Hence there is $b \in R$ and $c \in \mathfrak{m}$ with $ab + c = 1$, so that $ab = 1 - c$. But now $1 - c \in 1 + \mathfrak{m}$, which by assumption only consists of units. Hence $ab$ is a unit. But then both factors must be units, so that $a$ must be a unit.
\par
We have shown that $R \setminus \mathfrak{m} \subseteq R^\times$, i.e. $\mathfrak{m} \supseteq R \setminus R^\times$. But any proper ideal only consists of non-units, so that $\mathfrak{m} \subseteq R \setminus R^\times$. Hence $\mathfrak{m} = R \setminus R^\times$, which means exactly that $R$ is local with unique maximal ideal $\mathfrak{m}$.
}
\item {
Let $\overline{\mathfrak{m}} \subseteq \quotient{\mathbb{Q}[x, y]}{\langle x^{20}, y^{20} \rangle}$ be a maximal ideal. This corresponds to a maximal ideal $\mathfrak{m} \subseteq \mathbb{Q}[x, y]$ with $\langle x^{20}, y^{20} \rangle \subseteq \mathfrak{m}$. Hence $x^{20}, y^{20} \in \mathfrak{m}$. But since $\mathfrak{m}$ is a maximal ideal, it's also a prime ideal. Hence $x, y \in \mathfrak{m}$, hence $\langle x, y \rangle \subseteq \mathfrak{m}$. Since $\langle x, y \rangle \subseteq \mathbb{Q}[x, y]$ is a maximal ideal, we have $\langle x, y \rangle = \mathfrak{m}$. Hence $\quotient{\mathbb{Q}[x, y]}{\langle x^{20}, y^{20} \rangle}$ has a unique maximal ideal, i.e. it's a local ring.
}
\item {
One can see that the map
\[
\begin{array}{>{\displaystyle}c}
\quotient{\mathbb{C}[x, y]}{\langle x^3 - y^5 \rangle} \to \mathbb{C}[t^3, t^5], \\ \relax
[x] \mapsto t^5, \\ \relax
[y] \mapsto t^3 \\
\end{array}
\]
is an isomorphism.
\par
Clearly, $\mathbb{C}[t^3, t^5]$ is an integral domain but not a field. Hence $\langle x^3 - y^5 \rangle \subseteq \mathbb{C}[x, y]$ is a prime ideal but not a maximal ideal.
}
\end{enumerate}
}
\item {
\begin{enumerate}[label={(\alph*)}]
\item {
Let $M$ and $N$ be $R$-modules. Then the tensor product $M \otimes_R N$ is an $R$-module equipped with a bilinear map $\alpha: M \times N \to M \otimes_R N$ such that for any $R$-module $P$ and any bilinear map $\beta: M \times N \to P$ there is a unique linear map $\tilde{\beta}: M \otimes_R N \to P$ with $\beta = \tilde{\beta} \circ \alpha$.
\begin{center}
% https://tikzcd.yichuanshen.de/#N4Igdg9gJgpgziAXAbVABwnAlgFyxMJZABgBpiBdUkANwEMAbAVxiRAFkACAHW7wFt4nAHIgAvqXSZc+QigCM5KrUYs2XXhAHwA+gCUR4ySAzY8BIovnL6zVohAAFccphQA5vCKgAZgCcIfiQyEBwIJEUVOzZeRjQACzojXwCgxBCwpAAmals1B14AIxgcJOoGOmKGR2lzORA-LHd4nGSQf0CI6kzEHKj8kF48BlhgIpK6MRcxIA
\begin{tikzcd}
M \times N \arrow[r, "\alpha"] \arrow[rd, "\beta"'] & M \otimes_R N \arrow[d, "\tilde{\beta}"] \\
                                                    & P                                       
\end{tikzcd}
\end{center}
}
\item {
Take any element of $U^{-1} R \otimes_R M$. Such an element can be written as $\sum_{i = 1}^n \frac{a_i}{b_i} c_i$ with $a_i \in R$, $b_i \in U$ and $c_i \in M$. But now
\[
\sum_{i = 1}^n \frac{a_i}{b_i} c_i = \sum_{i = 1}^n \frac{\left( \prod_{j \neq i} b_j \right) a_i}{\prod_{j} b_j} c_i = \frac{1}{\prod_{j} b_j} \sum_{i = 1}^n \left( \prod_{j \neq i} b_j \right) a_i c_i
\]
with $\prod_{j} b_j \in U$ and $\sum_{i = 1}^n \left( \prod_{j \neq i} b_j \right) a_i c_i \in M$.
}
\item {
We have
\[
\begin{array}{>{\displaystyle}c>{\displaystyle}l}
& \mathbb{Q} \otimes_{\mathbb{Z}} \left( \mathbb{Z} \oplus \quotient{\mathbb{Z}}{\langle 42 \rangle} \right) \\
\cong & \left( \mathbb{Q} \otimes_{\mathbb{Z}} \mathbb{Z} \right) \oplus \left( \mathbb{Q} \otimes_{\mathbb{Z}} \quotient{\mathbb{Z}}{\langle 42 \rangle} \right) \\
\cong & \mathbb{Q} \oplus 0 \\
\cong & \mathbb{Q} \\
\end{array}
\]
}
\item {
It is a well-known fact from linear algebra that every module over a field is free.
\par
On the other hand, assume that every $R$-module is free. Now take $a \in R$ with $a \neq 0$. Then, by assumption, the $R$-module $\quotient{R}{\langle a \rangle}$ is free. If it is free of nonzero rank, multiplication by $a$ must be a nonzero endomorphism of it. However, multiplication by $a$ is just the zero endomorphism. Hence it must be free of rank zero, i.e. $\quotient{R}{\langle a \rangle} \cong 0$, so that $\langle a \rangle = R$, i.e. $a$ is a unit. Hence every nonzero element of $R$ is invertible, so that $R$ is a field.
}
\item {
Assume that $R$ is an integral domain.
\par
If $R$ is even a field, then every $R$-module is free and hence flat.
\par
On the other hand, assume that every $R$-module is flat. Now take $a \in R$ with $a \neq 0$. Now since $R$ is an integral domain, the map $R \overset{\cdot a}{\to} R$ is an injective endomorphism of $R$. Since every $R$-module is flat, $\quotient{R}{\langle a \rangle}$ is flat, so that $\quotient{R}{\langle a \rangle} \overset{\cdot a}{\to} \quotient{R}{\langle a \rangle}$ must also be injective. However, multiplication by $a$ is just the zero endomorphism on $\quotient{R}{\langle a \rangle}$. Hence we must have $\quotient{R}{\langle a \rangle} \cong 0$, so that $\langle a \rangle = R$. Hence $a$ is a unit. We have shown that every nonzero element of $R$ is invertible, i.e. that $R$ is a field.
}
\end{enumerate}
}
\item {
\begin{enumerate}[label={(\alph*)}]
\item {
A ring $R$ is called Artinian if it satisfies the descending chain condition:
\par
Every infinite descending chain of ideals
\[
I_1 \supseteq I_2 \supseteq I_3 \supseteq \ldots
\]
eventually stabilizes, i.e. there is $n \in \mathbb{N}$ with $\forall m \geq n : I_m = I_n$.
}
\item {
First of all, we will show that every Artinian domain is a field.
\par
Let $R$ be an Artinian domain. Take $a \in R$ with $a \neq 0$. Then we have a descending chain
\[
R \supseteq \langle a \rangle \supseteq \langle a^2 \rangle \supseteq \ldots
\]
This chain must eventually stabilize. Hence there is $n \in \mathbb{N}$ with $\langle a^n \rangle = \langle a^{n + 1} \rangle$. In particular, we have $a^n \in \langle a^{n + 1} \rangle$. This means that there is $b \in R$ with $a^n = b a^{n + 1}$, hence $0 = b a^{n + 1} - a^n = a^n \left( b a - 1 \right)$. Since $a \neq 0$ and $R$ is a domain, we must in fact have $0 = b a - 1$, hence $1 = b a$. Hence $a$ is a unit. Since every nonzero unit of $R$ is invertible, $R$ is a field.
\par
Now we will show that in an Artinian ring every prime ideal is maximal.
\par
Let $R$ be Artinian and $\mathfrak{p} \subseteq R$ be a prime ideal. Then $\quotient{R}{\mathfrak{p}}$ is an Artinian domain. By the first part, $\quotient{R}{\mathfrak{p}}$ is in fact a field. Hence $\mathfrak{p} \subseteq R$ is in fact a maximal ideal.
}
\item {
We can compute
\[
\begin{array}{>{\displaystyle}c>{\displaystyle}l}
& \langle xy, xz, yz \rangle \\
= & \langle x, xz, yz \rangle \cap \langle y, xz, yz \rangle \\
= & \langle x, yz \rangle \cap \langle y, xz \rangle \\
= & \langle x, y \rangle \cap \langle x, z \rangle \cap \langle y, x \rangle \cap \langle y, z \rangle \\
= & \langle x, y \rangle \cap \langle x, z \rangle \cap \langle y, z \rangle
\end{array}
\]
All of these are clearly prime. Hence we have found a primary decomposition of $\langle xy, xz, yz \rangle$.
}
\item {
}
\item {
}
\end{enumerate}
}
\end{enumerate}
\end{document}