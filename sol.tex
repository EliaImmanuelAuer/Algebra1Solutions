\documentclass{article}

\usepackage[a4paper, margin=2cm]{geometry}
\usepackage{microtype}
\usepackage{amsmath}
\usepackage{amssymb}
\usepackage{enumitem}

\newcommand\quotient[2]{{^{\displaystyle #1}}/{_{\displaystyle #2}}}

\begin{document}
\begin{enumerate}[style=nextline,label={Problem (\arabic*)}]
\item {
\begin{enumerate}[label={(\alph*)}]
\item {
Let $P = \{I \subsetneq R \mid I \text{ is an ideal of } R\}$ be the partially ordered set of proper ideals of $R$. Then $\mathfrak{m} \in P$ is called a maximal ideal if it is a maximal element of this partially ordered set.
\par
Equivalently, this means that $\quotient{R}{\mathfrak{m}}$ is a field.
}
\item {
Let $a \not\in \mathfrak{m}$. Then $\langle a \rangle + \mathfrak{m} \supsetneq \mathfrak{m}$, so $\langle a \rangle + \mathfrak{m} = R$. Hence there is $b \in R$ and $c \in \mathfrak{m}$ with $ab + c = 1$, so that $ab = 1 - c$. But now $1 - c \in 1 + \mathfrak{m}$, which by assumption only consists of units. Hence $ab$ is a unit. But then both factors must be units, so that $a$ must be a unit.
\par
We have shown that $R \setminus \mathfrak{m} \subseteq R^\times$, i.e. $\mathfrak{m} \supseteq R \setminus R^\times$. But any proper ideal only consists of non-units, so that $\mathfrak{m} \subseteq R \setminus R^\times$. Hence $\mathfrak{m} = R \setminus R^\times$, which means exactly that $R$ is local with unique maximal ideal $\mathfrak{m}$.
}
\item {
}
\item {
}
\end{enumerate}
}
\item {
\begin{enumerate}[label={(\alph*)}]
\item {
}
\item {
}
\item {
}
\item {
}
\item {
}
\end{enumerate}
}
\item {
\begin{enumerate}[label={(\alph*)}]
\item {
}
\item {
}
\item {
}
\item {
}
\item {
}
\end{enumerate}
}
\end{enumerate}
\end{document}